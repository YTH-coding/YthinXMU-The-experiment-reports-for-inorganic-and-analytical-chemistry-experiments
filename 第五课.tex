\documentclass[a4paper,12pt]{article}
\usepackage{geometry} %设置边距,符合Word设定
\geometry{a4paper, top=2.5cm, bottom=2.5cm, left=2.5cm, right=2.5cm}
\usepackage{amsmath} %数学公式排版
\usepackage{lmodern} %使用Latin Modern字体
\usepackage{multirow} %单元格合并
\usepackage{makecell} %优化单元格显示
\usepackage{tabularx} %Latex的表格
\usepackage{xeCJK} %排版中日韩(CJK)文字
\usepackage{fontspec} %提供了一个自动和统一的接口来加载字体,\setmainfont用于设置主字体
\usepackage{ctex} %中文支持
\usepackage{cite} %文献引用
\usepackage{pgfplots} %绘制散点图等图像
\pgfplotsset{compat=1.16}
\usepackage[version=4]{mhchem} %排版化学公式和方程式
\usepackage{chemfig} %排版化学结构式
\setchemfig{atom sep=2em}
\usepackage{xcolor} %在文档中为文字等添加颜色

\setmainfont{Times New Roman}

% 导言区
\title{\heiti\zihao{2} 实验:电离常数和电离度的测定}
\author{专业名称\quad 姓\;名\quad 学号xxxxxxxxxxxxxx}
\date{20xx年xx月xx日}

%设置小四号字号,字号为12pt,行距为18pt
\renewcommand{\normalsize}{\fontsize{12pt}{18pt}\selectfont}
%文件开始
\begin{document}

\maketitle

\setcounter{section}{0}
\section*{一、实验原理与操作方法}
\subsection*{(一)实验原理}
乙酸($\rm CH_3COOH$或HAc)是弱电解质,在水溶液中存在下列电离平衡:
$$
\ce{HAc <=> H^+ + Ac^-}
$$
$$
K_a=\frac{[H^+][Ac^-]}{[HAc]}\quad,\quad \alpha = \frac{[H^+]}{c}
$$
式中:$\rm [H^+]$、$\rm [Ac^-]$和$\rm [HAc]$分别为$\rm H^+$、$\rm Ac^-$和$\rm HAc$的平衡浓度,$K_a$是电离平衡常数,$\alpha$是电离度。

HAc溶液的总浓度c可以使用NaOH标准溶液滴定测得;由电离平衡式,可知$\rm H^+$和$\rm Ac^-$的平衡浓度相等,使用pH计测量溶液的pH,再根据$\rm pH=-lg[H^+]$就可以得到平衡浓度;
另外,[HAc]=c-$\rm [H^+]$,带入公式可以计算得到该温度下的$\rm K_a$和$\alpha$。

\subsection*{(二)操作方法}
(1)使用分析天平取三份邻苯二甲酸氢钾($\rm KHC_8H_4O_4$)固体于锥形瓶中,质量区间为$\rm 0.4\sim0.6g$,加入$\rm 40\sim60ml$纯水使其溶解,并加入$\rm 2\sim3$滴酚酞指示剂;

(2)取15ml的NaOH溶液,稀释至约300mL,用该溶液滴定锥形瓶中的$\rm KHC_8H_4O_4$溶液,使其变为微红色,并且保持半分钟不褪色(滴定三次);

(3)根据$\rm KHC_8H_4O_4$的质量和消耗NaOH溶液的体积计算NaOH溶液的浓度、相对偏差、平均相对偏差;

(4)取25mL的HAc溶液置于锥形瓶中,滴加酚酞指示剂之后用标定好的NaOH溶液进行滴定,计算得出HAc溶液的浓度、相对偏差、平均相对偏差(滴定三次);

(5)分别取5mL、10mL、25mL已标定浓度的HAc溶液于三个不同的50mL容量瓶中,定容至刻度线处,此时可以得到四种不同浓度的HAc溶液;

(6)用标准缓冲溶液校准pH计,将上述的四种不同浓度的HAc溶液注入50mL的小烧杯中,然后用校准后的pH计由稀到浓依次测量它们的pH值,并记录温度。

\subsection*{(三)实验中待记录的数据}
$\rm KHC_8H_4O_4$的质量、标定NaOH时所消耗的NaOH溶液的体积、标定HAc时所消耗的NaOH溶液的体积、4种不同浓度HAc溶液的pH值、测量pH时的温度。

\subsection*{(四)化学反应方程式}
$$
\rm\chemfig{*6(-=(-[7]COOH)-(-[1]COOK)=-=)}+NaOH=\chemfig{*6(-=(-[7]COONa)-(-[1]COOK)=-=)}+H_2O
$$

$$
\ce{HAc <=> H^+ + Ac^-}
$$

\section*{二、结果与讨论}

\subsubsection*{表1:NaOH溶液标定及HAc溶液浓度测定数据表}
\renewcommand\arraystretch{1}
\fontsize{10pt}{12pt}\selectfont
\noindent
\begin{tabularx}{13cm}{|p{0.5cm}|p{5cm}|p{1.8cm}|p{1.9cm}|p{1.9cm}|p{1.9cm}|}
    \cline{1-6}
    \multirow{8}{*}{\makecell{标\\ \\ \\定}}
        & \multicolumn{2}{c|}{编号} & \makecell{1} & \makecell{2} & \makecell{3}\\
        \cline{2-6}
        & \multicolumn{2}{c|}{基准物邻苯二甲酸氢钾质量m/g} & \makecell{0.5163} & \makecell{0.4311} & \makecell{0.5236}\\
        \cline{2-6}
        & \multicolumn{2}{c|}{所用NaOH溶液体积V/mL} & \makecell{23.82} & \makecell{19.92} & \makecell{24.18}\\
        \cline{2-6}
        & \multicolumn{2}{c|}{全距/mL} & \multicolumn{3}{c|}{0.36} \\
        \cline{2-6}
        & \multirow{2}{*}{\makecell{\ \ \ \ NaOH溶液浓度$\rm C/mol\cdot L^{-1}$\\}}
            & \makecell{测定值} & \makecell{0.1061} & \makecell{0.1060} & \makecell{0.1060}\\
            \cline{3-6}
            & & \makecell{平均值} & \multicolumn{3}{c|}{0.1060}\\
        \cline{2-6}
        & \multicolumn{2}{c|}{NaOH溶液浓度测定值的相对偏差} & \makecell{$\rm 0.09\%$} & \makecell{0} & \makecell{0}\\
        \cline{2-6}
        & \multicolumn{2}{c|}{NaOH溶液浓度测定值的平均相对偏差} & \multicolumn{3}{c|}{$\rm 0.03\%$}\\
    \cline{1-6}
    \multirow{7}{*}{\makecell{测\\ \\ \\定}}
        & \multicolumn{2}{c|}{所取HAc溶液体积V/mL} & \makecell{25.00} & \makecell{25.00} & \makecell{25.00}\\
        \cline{2-6}
        & \multicolumn{2}{c|}{所用NaOH溶液体积V/mL} & \makecell{32.90} & \makecell{32.95} & \makecell{33.00}\\
        \cline{2-6}
        & \multicolumn{2}{c|}{全距/mL} & \multicolumn{3}{c|}{0.10} \\
        \cline{2-6}
        & \multirow{2}{*}{\makecell{\ \ \ \ HAc溶液浓度$\rm C/mol\cdot L^{-1}$\\}}
            & \makecell{测定值} & \makecell{0.1395} & \makecell{0.1397} & \makecell{0.1399}\\
            \cline{3-6}
            & & \makecell{平均值} & \multicolumn{3}{c|}{0.1397}\\
        \cline{2-6}
        & \multicolumn{2}{c|}{HAc溶液浓度测定值的相对偏差} & \makecell{$\rm -0.15\%$} & \makecell{0} & \makecell{$\rm 0.15\%$}\\
        \cline{2-6}
        & \multicolumn{2}{c|}{HAc溶液浓度测定值的平均相对偏差} & \multicolumn{3}{c|}{$\rm 0.10\%$}\\
    \cline{1-6}
\end{tabularx}

\bigskip
\normalsize

\subsubsection*{表2:测定醋酸电离度和电离常数的数据及处理(温度:$\rm 23.0^{\circ}C$)}
\renewcommand\arraystretch{1.2}
\noindent
\fontsize{10pt}{12pt}\selectfont
\begin{tabularx}{13.05cm}{|p{0.75cm}|p{1.6cm}|p{2cm}|p{2.4cm}|p{1cm}|p{2.3cm}|p{2cm}|p{1cm}|}
    \cline{1-8}
    \makecell{编号} & \makecell{$\rm V_{HAc}/mL$} & \makecell{定容体积/mL} & \makecell{$\rm C_{HAc}/mol\cdot L^{-1}$} & \makecell{pH} & \makecell{$\rm [H^+]/mol\cdot L^{-1}$} & \makecell{$\rm K_a$} & \makecell{$\alpha$}\\
    \cline{1-8}
    \makecell{1} & \makecell{5.00} & \makecell{50.00} & \makecell{0.01397} & \makecell{3.33} & \makecell{$4.7\times 10^{-4}$} & \makecell{$1.6\times 10^{-5}$} & \makecell{$3.4\%$}\\
    \cline{1-8}
    \makecell{2} & \makecell{10.00} & \makecell{50.00} & \makecell{0.02794} & \makecell{3.15} & \makecell{$7.1\times 10^{-4}$} & \makecell{$1.8\times 10^{-5}$} & \makecell{$2.4\%$}\\
    \cline{1-8}
    \makecell{3} & \makecell{25.00} & \makecell{50.00} & \makecell{0.06985} & \makecell{2.92} & \makecell{$1.2\times 10^{-3}$} & \makecell{$2.1\times 10^{-5}$} & \makecell{$1.7\%$}\\
    \cline{1-8}
    \makecell{4} & \makecell{50.00} & \makecell{------} & \makecell{0.1397} & \makecell{2.78} & \makecell{$1.7\times 10^{-3}$} & \makecell{$2.0\times 10^{-5}$} & \makecell{$1.2\%$}\\
    \cline{1-8}
\end{tabularx}

\bigskip

\normalsize
经过计算,$\rm 23.0^{\circ}C$下醋酸的电离平衡常数$\rm K_a$的平均值为:$1.9\times 10^{-5}$,与文献值符合程度较好。

若以298K时的电离平衡常数\textsuperscript{\cite{B1}}为基准(在该温度附近,$K_a$随温度的变化不大\textsuperscript{\cite{JJSZ198706006}}),四种浓度下电离平衡常数测定值的相对偏差、平均相对偏差、相对误差、平均相对误差如下表:

\medskip
\fontsize{10pt}{12pt}\selectfont
\renewcommand\arraystretch{1.2}
\begin{tabularx}{12.3cm}{|p{3.5cm}|p{2.2cm}|p{2.2cm}|p{2.2cm}|p{2.2cm}|}
    \cline{1-5}
    \makecell{编号} & \makecell{1} & \makecell{2} & \makecell{3} & \makecell{4}\\
    \cline{1-5}
    \makecell{$\rm C_{HAc}/mol\cdot L^{-1}$} & \makecell{0.01397} & \makecell{0.02794} & \makecell{0.06985} & \makecell{0.1397}\\
    \cline{1-5}
    \makecell{$K_a$} & \makecell{$1.6\times 10^{-5}$} & \makecell{$1.8\times 10^{-5}$} & \makecell{$2.1\times 10^{-5}$} & \makecell{$2.0\times 10^{-5}$}\\
    \cline{1-5}
    \makecell{$K_a$平均值} & \multicolumn{4}{c|}{$1.9\times 10^{-5}$}\\
    \cline{1-5}
    \makecell{相对偏差} & \makecell{$\rm -16\%$} & \makecell{$\rm -6\%$} & \makecell{$\rm 11\%$} & \makecell{$\rm 6\%$}\\
    \cline{1-5}
    \makecell{平均相对偏差} & \multicolumn{4}{c|}{$10\%$}\\
    \cline{1-5}
    \makecell{298K下的$K_a$} & \multicolumn{4}{c|}{$1.75\times 10^{-5}$}\\
    \cline{1-5}
    \makecell{相对误差} & \makecell{$\rm -9\%$} & \makecell{$\rm 3\%$} & \makecell{$\rm 20\%$} & \makecell{$\rm 15\%$}\\
    \cline{1-5}
    \makecell{平均相对误差} & \multicolumn{4}{c|}{$8\%$}\\
    \cline{1-5}
\end{tabularx}

\normalsize

\section*{三、实验分析与总结}
\subsection*{(一)、实验分析}
本实验通过pH法测定醋酸溶液的pH值(在校准pH计时,仪表显示的斜率为100.4)、通过酸碱滴定测定醋酸的浓度c来计算在一定温度下醋酸的电离平衡常数$K_a$与电离度$\alpha$,实验原理较为简单。

通过该实验,我们可以得到以下几个结论:

(1)对于弱酸,其浓度越小,其电离度越大;

(2)浓度越大,弱酸溶液的$\rm [H^+]$越大,pH值越小;

(3)在不同浓度下,电离平衡常数略有波动,但从变化趋势来看(具体可看下图),这种波动应当是由实验误差造成的,电离平衡常数的大小与弱酸的浓度无关。

\begin{tikzpicture}
    \begin{axis}[
        width=14cm, % 增加图表的宽度
        height=6cm, % 可以同时设置高度
        title={图1. $\quad K_a -C_{HAc}$},
        xlabel={$C_{HAc}\times 0.1mol\cdot L^{-1}$},
        ylabel={$K_a \times 10^{-5}$},
        xmin=0, xmax=1.5,
        ymin=0, ymax=2.2,
        xtick={0, 0.2, 0.4, 0.6, 0.8, 1.0, 1.2, 1.4},
        ytick={0.2, 0.4, 0.6, 0.8, 1.0, 1.2, 1.4, 1.6, 1.8, 2.0 },
        axis lines=left, % 坐标轴从左下角开始绘制
        grid=none, % 移除网格线
        major tick length=4pt, % 设置刻度线长度
        x axis line style={->, line width=1pt}, % 为 x 轴添加单侧箭头
        y axis line style={->, line width=1pt}, % 为 y 轴添加单侧箭头
        xtick align=inside, % 将 x 轴的刻度线放置在外侧
        ytick align=inside, % 将 y 轴的刻度线放置在外侧
        tick style={line width=1pt, color=black},
        font=\footnotesize
    ]
    \addplot[color=blue, line width=1pt,]
        coordinates {(0.1397, 1.6)(0.2794, 1.8)(0.6985, 2.1)(1.397, 2.0)};
    \addplot[only marks, mark=*, mark size=2pt, color=blue]
        coordinates {(0.1397, 1.6)(0.2794, 1.8)(0.6985, 2.1)(1.397, 2.0)};
    \end{axis}
\end{tikzpicture}


\bigskip

查阅文献,醋酸在$\rm 298K$下的电离平衡常数为$\rm 1.75\times 10^{-5}$\textsuperscript{\cite{B1}},在$\rm 25.0^{\circ}C$附近,电离平衡常数$K_a$与温度之间关联不明显,
图像较为“平坦”,平坦段出现在$\rm 20^{\circ}C\sim 24^{\circ}C$\textsuperscript{\cite{JJSZ198706006}},于是可以认为在$\rm 23.0^{\circ}C$的温度下,电离平衡常数约为$\rm 1.75\times 10^{-5}$,
而本实验的测定值与其略有偏差。出现该误差,原因可能是:

(1)滴定时出现操作误差;

(2)醋酸易挥发,如果暴露在空气中太久会导致浓度的偏差;

(3)转移溶液的时候,容器内壁有水,降低了溶液浓度。

\subsection*{(二)、实验改进}
依据误差的来源,本实验有以下的几个改进方向:

(1)醋酸易挥发,NaOH容易和空气中的$\rm CO_2$发生反应,因此需要快速进行试验,尽量减少因挥发或者吸收空气中的物质导致的浓度变化;

(2)测定pH时,盛放溶液的容器在盛放前应洁净干燥,使用滤纸片擦干水分;

(3)进行滴定操作,快到滴定终点时需要放慢滴加溶液的速度,避免溶液添加过多。

\section*{四、思考题}

(1)弱电解质溶液的电离度($\alpha$)与哪些因素有关?

弱电解质的电离度与弱电解质溶液的温度、浓度以及弱电解质的电离平衡常数有关。

(2)若分别改变HAc溶液的温度和浓度,测得的$K_a$和$\alpha$有无变化?为什么?

改变温度时,测得的$K_a$和$\alpha$均有变化,因为两者都和温度有关;
但若是仅改变浓度,由于$K_a$只和温度有关,$K_a$将不会发生改变,但是会改变$\alpha$,因为:
$$
\alpha=\frac{[H^+]}{c}\quad,\quad K_a=\frac{[H^+]{[Ac^-]}}{[HAc]}=\frac{[H^+]^2}{c-[H^+]}
$$
根据两个式子,可以计算得到:
$$
[H^+]=\frac{-K_a+\sqrt{K_a^2+4K_ac}}{2}
$$$$
\alpha=\frac{[H^+]}{c}=\frac{-K_a+\sqrt{K_a^2+4K_ac}}{2c}=-\frac{K_a}{2c}+\sqrt{(\frac{K_a}{2c})^2+\frac{K_a}{c}}
$$
根据该式子,容易可以看出浓度c越小,电离度$\alpha$越大。

\bibliographystyle{IEEEtran}
\bibliography{data_5}

\end{document}