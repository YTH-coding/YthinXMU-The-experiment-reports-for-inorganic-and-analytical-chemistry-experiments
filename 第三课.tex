\documentclass[a4paper,12pt]{article}
\usepackage{geometry}%设置边距,符合Word设定
\geometry{a4paper, top=2.5cm, bottom=2.5cm, left=2.5cm, right=2.5cm}
\usepackage{amsmath} %数学公式排版
\usepackage{lmodern} %使用Latin Modern字体
\usepackage{multirow} %单元格合并
\usepackage{makecell} %优化单元格显示
\usepackage{tabularx} %Latex的表格
\usepackage{xeCJK} %排版中日韩(CJK)文字
\usepackage{fontspec} %提供了一个自动和统一的接口来加载字体,\setmainfont用于设置主字体
\usepackage{ctex} %中文支持

\setmainfont{Times New Roman}

% 导言区
\title{\heiti\zihao{2} 实验:滴定分析基本操作练习}
\author{专业名称\quad 姓\;名\quad 学号xxxxxxxxxxxxxx}
\date{20xx年xx月xx日}

% 设置小四号字号,字号为12pt,行距为18pt
\renewcommand{\normalsize}{\fontsize{12pt}{18pt}\selectfont}

%文件开始
\begin{document}

\maketitle

\setcounter{section}{0}
\section*{一、实验原理与操作方法}
\subsection*{(一)实验原理}
本实验采用酸碱滴定法,以酸碱反应为基础。水溶液中的酸碱反应是中和反应,其离子方程式为:
$$
\rm
H_{3}O^++OH^-=2H_2O
\quad ; \quad
H_{3}O^++A^-=HA+H_2O
$$
酸碱中和反应的当等点pH=7.00,在该点附近,微量的酸或碱就可以导致溶液pH的突变(即:滴定突越)。
于是,若使用变色范围处于滴定突跃范围内的酸碱指示剂,就可以在滴定终点附近见到由于pH突变导致的溶液颜色变化。

\subsection*{(二)操作方法}
(1)准备器材:准备好一支聚四氟乙烯滴定管、三个锥形瓶、两个烧杯、一支移液管,还有用于盛装NaOH溶液和HCl溶液的瓶子;

(2)溶液配制:各取25mL浓度均为2mol/L的NaOH溶液和HCl溶液,加水稀释至500mL,盖上盖子备用;

(3)仪器清洗:用自来水清洗滴定管,然后用纯水润洗滴定管3次,再用待加入液体(比如配置好的NaOH溶液)润洗3次,搁置备用;
之后对移液管进行如上操作,但是使用另一种溶液(比如配置好的HCl溶液)润洗;

(4)加注溶液:将对应于滴定管的溶液(比如配置好的NaOH溶液)加注进滴定管,排除气泡并将液面调零;用移液管移取3次另一种溶液至3个锥形瓶中,
加入酸碱指示剂(这里使用酚酞);

(5)滴定:用滴定管滴加溶液,直至最后半滴落下锥形瓶内溶液恰好变色,垂放滴定管,记录此时滴定管内的液面;

(6)重复操作:换用另一个锥形瓶,向滴定管中注入溶液并调零之后继续滴定,
3次结束之后,对滴定管、移液管、锥形瓶清洗,并对前两者用纯水润洗,更换溶液后用对应的溶液润洗,更换指示剂(这里换用甲基橙),继续滴定。

\subsection*{(三)实验中测定或标定的数据}
(1)NaOH溶液滴定HCl溶液:HCl溶液的体积${\rm V_{HCl}(mL)}$;
滴定所消耗的NaOH溶液的体积${\rm V_{NaOH}(mL)}$。

(2)HCl溶液滴定NaOH溶液:NaOH溶液的体积${\rm V_{NaOH}(mL)}$;
滴定所消耗的HCl溶液的体积${\rm V_{HCl}(mL)}$。

\subsection*{(四)反应方程式}
$$
\rm
H_{3}O^++OH^-=2H_2O
\quad ; \quad
H_{3}O^++A^-=HA+H_2O
$$

\section*{二、结果与讨论}

\subsection*{表1.NaOH溶液滴定HCl溶液(指示剂:酚酞)}

\renewcommand\arraystretch{1.5}

\begin{tabularx}{13cm}{|p{4cm}|p{3cm}|p{3cm}|p{3cm}|}
    \cline{1-4}
    \makecell{记录与计算\\} & \makecell{1} &  \makecell{2} &  \makecell{3} \\
    \cline{1-4}
    \makecell{$\rm V_{HCl}(mL)$\\} & \makecell{25.00} & \makecell{25.00} & \makecell{25.00} \\ 
    \cline{1-4}
    \makecell{$\rm V_{NaOH}(mL)$\\} & \makecell{21.05} & \makecell{21.00} & \makecell{21.02} \\ 
    \cline{1-4}
    \makecell{$\rm \overline{V}_{NaOH}(mL)$\\} & \multicolumn{3}{|c|}{21.02} \\ 
    \cline{1-4}
    \makecell{全距 (mL)\\} & \multicolumn{3}{|c|}{0.05} \\ 
    \cline{1-4}
    \makecell{${\rm V_{HCl}(mL)}$\\$\rm \overline{{\overline{V}_{NaOH}(mL)}}$} & \multicolumn{3}{|c|}{1.189} \\ 
    \cline{1-4}
    \makecell{$\rm \text{相对偏差}d$\\} & \makecell{0.14\%} & \makecell{-0.10\%} & \makecell{0.00\%}\\
    \cline{1-4}
    \makecell{$\rm \text{平均相对偏差}\overline{d}$\\} & \multicolumn{3}{|c|}{0.08\%}\\
    \cline{1-4}
\end{tabularx}

\subsection*{表2.HCl溶液滴定NaOH溶液(指示剂:甲基橙)}
    
\begin{tabularx}{13cm}{|p{4cm}|p{3cm}|p{3cm}|p{3cm}|}
    \cline{1-4}
    \makecell{记录与计算\\} & \makecell{1} &  \makecell{2} &  \makecell{3} \\ 
    \cline{1-4}
    \makecell{$\rm V_{NaOH}(mL)$\\} & \makecell{25.00} & \makecell{25.00} & \makecell{25.00} \\ 
    \cline{1-4}
    \makecell{$\rm V_{HCl}(mL)$\\} & \makecell{29.72} & \makecell{29.75} & \makecell{29.70} \\ 
    \cline{1-4}
    \makecell{$\rm \overline{V}_{HCl}(mL)$\\} & \multicolumn{3}{|c|}{29.72} \\ 
    \cline{1-4}
    \makecell{全距 (mL)\\} & \multicolumn{3}{|c|}{0.05} \\ 
    \cline{1-4}
    \makecell{$\rm \underline{{V_{NaOH}(mL)}}$\\${\rm \overline{V}_{HCl}(mL)}$} & \multicolumn{3}{|c|}{0.9242} \\ 
    \cline{1-4}
    \makecell{$\rm \text{相对偏差}d$\\} & \makecell{0.00\%} & \makecell{-0.10\%} & \makecell{0.07\%}\\
    \cline{1-4}
    \makecell{$\rm \text{平均相对偏差}\overline{d}$\\} & \multicolumn{3}{|c|}{0.06\%}\\
    \cline{1-4}
\end{tabularx}

\bigskip
对于HCl溶液滴定NaOH溶液:
$$
\rm \overline{V}_{HCl}=\frac{1}{3}\sum V_{HCl}=\frac{1}{3}(29.72+29.75+29.70)mL=29.72 mL
$$$$
\rm \text{全距}=29.75mL-29.70mL=0.05mL
$$$$
\rm {{V_{NaOH}}}/{\rm \overline{V}_{HCl}}=25.00/29.72=0.9242
$$$$
\rm d=\frac{V_{HCl}-\overline{V}_{HCl}}{\overline{V}_{HCl}}=\frac{29.72-29.72}{29.72}=0.00\%
$$$$
\rm \overline{d}=\frac{1}{3}\sum \lvert d\rvert 
=\frac{1}{3}(\lvert 0.00\%\rvert + \lvert -0.10\%\rvert + \lvert 0.07\%\rvert)=0.06\%
$$

对于该组其他相对偏差的计算,以及NaOH溶液滴定HCl溶液实验中的数据处理,参考上面,同理可得。

\section*{三、实验分析与总结}
\subsection*{(一)实验分析}
本实验是化学中的经典定量分析实验,通过该实验,我们能够充分了解化学实验的魅力。
在操作规范的前提下,误差较小。但是因为操作不规范、读书出现偏差、细节被疏漏等问题,实验仍有改进之处。
\subsection*{(二)实验改进}
(1)在量取25mL浓度均为2mol/L的NaOH溶液和HCl溶液时,应当使用测量仪器(如:吸量管等)进行较为精确地量取;

(2)稀释溶液时,需要选择带有刻度线的大容量容器进行稀释,保证稀释过程的准确;

(3)在进行滴定操作时,应当戴上口罩或在通风橱中操作,避免口中呼出气体里的${\rm CO_2}$与NaOH反应,尤其是临近反应终点时;

(4)在滴定剂滴入时滴定剂滴入处变色,但是摇晃之后褪色,就表明即将到达滴定终点,此时需要缓慢滴加滴定剂甚至半滴半滴加入溶液。

(5)甲基红的变色范围更加靠近pH=7.00,也许在HCl溶液滴定NaOH溶液时,可以考虑使用甲基红。

\section*{四、思考题}
1、配置NaOH溶液时,应用何种天平称取试剂?为什么?

答:配置NaOH溶液时,应使用精确位数不至于过高的带托盘的天平,比如电子天平等,因为NaOH具有腐蚀性和吸湿性,
要求天平需要拥有托盘,如果精确位数过高,会导致无法读数。

2、在滴定分析中,滴定管为何要用滴定剂润洗几次?滴定中的锥形瓶是否也要用滴定剂润洗呢?为什么?

答:因为在用纯水润洗之后,滴定管上残留的水珠会降低后续进入滴定管内滴定剂的浓度,引起误差;
但是滴定用的锥形瓶不需要被滴定剂润洗,因为移液管加入的滴定剂的量是一定的,这就能够保证锥形瓶中溶质的量无误,其浓度变化对实验没有影响。

3、为了得到${\rm 0.1mol\cdot L^{-1}}$的HCl标准溶液,量取浓盐酸时应使用吸量管还是量筒,为什么?

答:应当使用吸量管,因为量筒的直径偏大,浓盐酸更易挥发,产生更大的误差,而吸量管开口较小。

4、通常情况下,如何清洗滴定管?

答:先将滴定管内的溶液放干净,再将滴定管取下,用清水清洗,用手封住开口,摇晃滴定管,使其流经管的内壁,以去除残留的液体;
可添加洗洁精等,让其在管内停留一段时间之后,用流水彻底冲洗滴定管,保证洗涤剂被全部冲出。

5、用待装入滴定管中的溶液润洗滴定管时,通常需洗几次?每次用量多少?

答:通常需要洗三次,每次需要使用大约10mL的待装入溶液。

\end{document}